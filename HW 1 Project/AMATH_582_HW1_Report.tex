\documentclass[11pt]{amsart}

% Standard letter size paper with 1inch margins
\usepackage[letterpaper, margin=1in]{geometry}

% Useful packages 
\usepackage{amsmath, amssymb, amsthm, amsaddr}
\usepackage{enumerate, subcaption, graphicx, hyperref}


\title{AMATH 482/582: Homework 1}
\author{Jonathan McCormack}
\address{Applied Mathematics Department, University of Washington, Seattle, WA 
\\jrmack20@uw.edu}
\date{\today}

\begin{document}

\maketitle 

\begin{abstract}
    Underwater movements of a submarine can be tracked if it emits a unique 
    frequency that influences the submarine's surrounding acoustic pressure. 
    Acoustic pressure data collected over 24 hours in the Puget Sound was analyzed 
    using the Fast Fourier Transform to identify the most dominant frequency present. 
    A Gaussian filter was applied to eliminate noise surrounding this center frequency and
    then the data was inverse transformed back into the spatial domain to plot the location 
    of the center frequency's source. This process can be incredibly useful in identifying key
    features of a system and its interactions with its surroundings.
\end{abstract}

\section{Introduction and Overview}\label{sec:Introduction}

Given 3-dimensional acoustic pressure data from a region of the Puget Sound over a 24-hour period, 
we intend to track the movements of an unknown submarine. We have no information about the submarine’s 
component equipment and its cumulative acoustic signature. However, the submarine’s signature frequency 
will influence the acoustic pressure underwater and will naturally propagate away from the submarine with 
decreasing amplitude. Therefore, identifying the location of the highest amplitude of the signature frequency 
will inherently locate the precise location of the submarine.\\

We intend to use the Fast Fourier Transform (FFT) to identify the submarine’s signature frequency. 
The FFT relates frequencies to their prominence (amplitude) in a N-dimensional space at a point in time. 
We can apply a Gaussian filter to remove unwanted frequencies (i.e., noise) 
from the FFT to produce a clean signal which only communicates information from the signature frequency. 
This clean signal can be inverse transformed to reveal the location of the submarine in the spatial domain. 

\section{Theoretical Background}\label{sec:Theory}

The Fourier Transform converts a continuous signal function of time into a function of frequency 
represented by a sum of sine and cosine basis functions\cite{kutzdata}. The Fourier Transform can be extended 
as needed to evaluate a signal in N-dimensional space. The Fourier Transform can also be 
inverted to return the amplitude as a function of time rather than as a function of frequency using 
the Inverse Fourier Transform. The below equations are the 1-dimensional 
forward\eqref{equ:forward-fourier-integral} and 
inverse\eqref{equ:inverse-fourier-integral} Fourier Transforms respectively.

\begin{equation}
    F(k) = \frac{1}{\sqrt{2\pi}} \int_{-\infty}^{\infty} f(t) e^{-\mathrm{i}kt} \mathrm{d}t
    \label{equ:forward-fourier-integral}
\end{equation}

\begin{equation}
    f(t) = \frac{1}{\sqrt{2\pi}} \int_{-\infty}^{\infty} F(k) e^{\mathrm{i}kt} \mathrm{d}k
    \label{equ:inverse-fourier-integral}
\end{equation}

Similarly, the Discrete Fourier Transform (DFT) applies the Fourier Transform to discrete signal date.
This is useful because unlike known signal functions that can be evaluated analytically, real-world data
always includes a discrete sampling rate. The below equations are the 1-dimensional\eqref{equ:discrete-fourier-transform}
and 3-dimensional\eqref{equ:3D-discrete-fourier-transform} DFT functions respectively.

\begin{equation}
    F[k] = \frac{1}{N} \sum_{n=0}^{N-1} f(x_n) e^{-\frac{\mathrm{i} 2 \pi k n}{N}} 
    \label{equ:discrete-fourier-transform}
\end{equation}

\begin{equation}
    F[K] = \frac{1}{N_xN_yN_z}\sum_{n_x=0}^{N_x-1}\sum_{n_y=0}^{N_y-1}\sum_{n_z=0}^{N_z-1} f(x_n,y_n,z_n) e^{-(\frac{\mathrm{i} 2 \pi k n_x}{N_x})-(\frac{\mathrm{i} 2 \pi k n_y}{N_y})}-(\frac{\mathrm{i} 2 \pi k n_z}{N_z}) 
    \label{equ:3D-discrete-fourier-transform}
\end{equation}

The FFT is the algorithm invented by John W. Tukey and James W. Cooley in the 1960s to quickly calculate the DFT. 
This is done by separating even signals from odd signals to reduce the total number of calculations. 
The FFT costs computations in the order of $Nlog(N)$, whereas the DFT costs significantly more computations 
in the order of $N^2$. Similarly to the Fourier Transform, the FFT can be inverted. After using the FFT, 
we reorder the even and odd signals by using the FFT shift function. FFT shift improves visualization by 
centering the FFT on the zero frequency.\\

The Gaussian filter is a simple function applied to data in the fourier domain to remove noise. It does this by 
applying a bell curve centered on the signal of interest $k_0$ and eliminating all frequencies outside of a 
standard deviation denoted by $\sigma$. To apply the Gaussian filter, the equation is solved for every value of 
k in the fourier domain mesh grid. Then the resulting array is multiplied element wise with the unfiltered FFT data.
The filtered data is unshifted with the IFFT shift and then inverse transformed with IFFT. 
This process is conducted at every time point independently. The below equations are the 
1-dimensional\eqref{eq:gaussian-filter} and 3-dimensional\eqref{eq:3D-gaussian-filter} Gaussian filters respectively.

\begin{equation}
      g(kx) = e^\frac{(kx-kx_0)^2}{2\sigma^2}
      \label{eq:gaussian-filter}
\end{equation}

\begin{equation}
      g(kx, ky, kz) = e^\frac{(kx-kx_0)^2+(ky-ky_0)^2+(kz-kz_0)^2}{2\sigma^2}
      \label{eq:3D-gaussian-filter}
\end{equation}

\section{Algorithm Implementation and Development}\label{sec:Algorithms}

The data was provided as a 262,144x49 array where each column was a flattened 64x64x64 vector representing
3-dimensional data at a single time point. The data was neither normalized nor did we receive information on 
the sensing equipments observable range. The code was written in Python using the pathlib, Numpy, Matplotlib, 
and Scipy libraries. The below pseudocode summarizes the general process of identifying the signature frequency, 
filtering the data using the Gaussian filter centered on the signature frequency, and locating the submarine by 
inverse transforming the filtered data at each time point.\\

\textbf{Pseudocode}
\begin{enumerate}
   \item Find the signature frequency:
   \begin{itemize}
     \item Reshape the flattened data for each time point into a 64x64x64 array
     \item Apply the N-dimensional FFT and FFT shift
     \item Sum the FFT arrays and find the time-averaged FFT
     \item Identify the most prominant frequencies in all 3 dimensions
   \end{itemize}
   \item Create the Gaussian Filter using the signature frequencies
   \item Identify the location of the submarine:
   \begin{itemize}
    \item Reshape the flattened data for each time point into a 64x64x64 array
    \item Apply the N-dimensional FFT and FFT shift
    \item Multiply the resultant array by the Gaussian filter array
    \item Apply the IFFT shift and the N-dimensional IFFT
    \item Identify the spatial position of the most prominant data point
   \end{itemize}
\end{enumerate}

\section{Computational Results}\label{sec:results}

Using the unfiltered, time-averaged data from the FFT, the signature frequency was detected 
\autoref{tab:Signature-frequencies}. The unfiltered data is noisy as shown below in the kz=0 
cross sections in \autoref{fig:Filtered-FFT-cross-section}. \autoref{fig:Filtered-FFT-cross-section} 
also includes the same cross section at kz=0 when the Gaussian filter is applied using $\sigma =$ 1, 2, and 3. 
Because the higher values of sigma obscure the central frequency, we used sigma = 1 for the Gaussian Filter 
when plotting the submarine’s path.\\

\begin{table}[h]
    \centering
    \begin{tabular}{|c|c|}
         \hline
         Coordinate plane & Signature frequency \\ \hline
         kx & -5.341 \\ 
         ky & -2.199 \\
         kz & 6.912 \\ \hline
    \end{tabular}
    \caption{Signature frequencies emitted by the submarine in the 3D frequency domain.}
    \label{tab:Signature-frequencies}
\end{table}

\begin{figure}[h]
    \centering
    \includegraphics[width=0.5\textwidth]{Figure1.png}
    \caption{Graphs of the time-averaged amplitude of FFT at kz=0. The top left graph is 
    unfiltered. The unfiltered graph has no clearly discernable peaks and clearly has noise.
    The remaining graphs apply Gaussian filters centered at the signature frequency
    using $\sigma =$ 1, 2, and 3.}
    \label{fig:Filtered-FFT-cross-section}
\end{figure}

Using the Inverse FFT on the filtered data at the first and last time point, 
we returned the following coordinate tuples.\\

\begin{table}[h]
    \centering
    \begin{tabular}{|c|c|c|}
         \hline
         Coordinate plane & Start point & End point \\ \hline
         x & 2.92 & -5.08 \\ 
         y & -7.85 & 6.00 \\
         z & -0.15 & 0.77 \\ \hline
    \end{tabular}
    \caption{Submarine start and end points}
    \label{tab:Submarine-Location}
\end{table}

The 3-dimensional path of the submarine during the 24-hour period is shown below in Figure 2. The path 
using the noisy (unfiltered) data is red; and the filtered path is green. There are two clearly 
visible outliers effectively removed by the Gaussian filter, showing its effectiveness at removing unwanted 
noise from the data.The x,y coordinates of the submarine are shown below in Figure 3. This 2-dimensional 
path provides aircraft the ability to overfly and potentially observe the submarine.\\

\begin{figure}[h]
    \centering
    \includegraphics[width=0.4\textwidth]{Figure2.png}
    \caption{3D submarine path using unfiltered (red) and filtered (green) data.}
    \label{fig:3D-Submarine-Path}
\end{figure}

\begin{figure}[h]
    \centering
    \includegraphics[width=0.4\textwidth]{Figure3.png}
    \caption{2D submarine path in the xy-plane showing the overhead view.}
    \label{fig:2D-Submarine-Path}
\end{figure}

\section{Summary and Conclusions}\label{sec:conclusions}

Given noisy 3-dimensional acoustic pressure data, the path of the submarine and the submarine's 
signature frequencies were clearly determined using the Fast Fourier Transform and the Gaussian filter.
The method eliminated clear outliers showing the effectiveness of the method.\\

Future improvements to the method should be focused on using knowledge about the sensing equipment to remove bad
data that is outside the sensitivity of the equipment and normalizing the data using baseline acoustic pressure information.
Additionally, the signature frequency should converted to Hertz adn be compared to known submarine signature frequencies to
inform data filtering methods. Lastly, a more robust filter should be added to separate direction frequencies (e.g., tail fins) 
from omnidirectional frequencies (e.g., communication equipment). This could improve the predictive nature of submarine
movement detection.

\section*{Acknowledgements}

I would like to thank my classmate Samuilad for our discussions about scientific computing and data analysis. 
I would also like to thank Professor Hosseini for his lectures, notes, and code examples.

\bibliographystyle{abbrv}
\bibliography{AMATH_582_HW1_references}

\end{document}
